\section{Competencias evidenciadas}

  El trabajo realizado por el alumno como ingeniero de \textit{software}, en Beetrack durante los cuatro meses evidencian de manera clara cada una de las tres competencias declaradas del perfil. En específico se pudo evidenciar lo siguiente:

  \begin{enumerate}
    \item \textbf{Aplicar conocimientos avanzados de Ciencia de la Computación, Ingeniería de Software y Sistemas de Información para entender problemas complejos y abiertos.}
    
    El estudiante desarrolló un sistema especializado con una arquitectura de software diseñada para ser escalable y replicable rápidamente mediante el uso de tecnologías como GKE y contenedores de Docker, como también hizo uso avanzado de del \textit{frmework} RoR utilizando validadores personalizados en los modelos. Adicionalmente, utilizó de manera amplia el \textit{framework} de pruebas RSpec, llevando buenas prácticas de \textit{software} al ERP de manera de poder eliminar el problema de la inseguridad que se tiene por parte del equipo al hacer cambios en el código.

    Por otro lado, para poder diseñar de manera adecuada los aparatos de \textit{software} creados, el alumno tuvo que aplicar conocimiento teórico de patrones avanzados de diseño, de manera de que se pudiese tener un código altamente cohesivo y con bajo acoplamiento, algo que es sumamente deseado. Para poder abordar los dos puntos mencionados anteriormente, el estudiante debió comprender las diferentes problemáticas del problema y utilizar sus conocimientos en las diferentes áreas.

    Finalmente, el estudiante utilizó conocimientos de sistemas de información, tales como la lógica de negocio de un ERP, que fue en lo que él trabajó. Esto se traduce directamente en la combinación e integración que se realiza con Hubspot y Quickbooks, junto con el uso que \textit{Hive} le da a nivel de lógica de negocio.

    \item \textbf{Diseñar y desarrollar modelos y artefactos de software y simular soluciones a problemas de la Ciencia e Ingeniería de Computación, cumpliendo con restricciones técnicas, sociales y éticas.}
    
    Por un lado esta competencia se puede ver evidenciada mediante el aumento sistemático del nivel de cobertura de código del ERP. Esto se debe a que las pruebas automatizadas son una forma de simulación de ejecución del código en un ambiente aislado, solucionando el problema de confiabilidad y resiliencia ante fallas. En este sentido, el alumno logró diseñar artefactos de \textit{software} que simularan la ejecución del mismo código escrito, manteniendo resguardada información sensible de otros entornos.

    Por otro lado, esta competencia también se pudo ver evidencada en las simulaciones de la emisión de pre-facturas y cálculo de uso por parte de los clientes. Esto se debe a que para poder tener seguridad que el servicio programado tenga una mayor precisión que los sistemas anteriores se debió realizar simulaciones de uso de manera de no itervenir los datos reales. A su vez, esto implicó restricciones sociales y éticas al no poder compartir la información de facturación real con nadie más, y al mismo tiempo se logró un incremento de casi un 50\% en términos de precisión de facturación.

    Adicionalmente, se desarrolló modelos y artefactos de \textit{software} mediante la creación de los modelos que el servicio de condiciones comerciales aloja (\texttt{plans}, \texttt{plan\_relations}, \texttt{plan\_relation\_proposals} y \texttt{comments}), como también de los artefactos que permiten el cálculo correcto de los montos acorde a los modelos generados. Cabe mencionar que los artefactos generados también cumplen con restricciones técnicas en cuanto al hecho de no mantener registros del resultado final, delegando esa tarea al ERP, dado que \textit{Hive} es el sistema que genera la factura real y mediante el cual se realizan los cobros a los clientes.

    \item \textbf{Analizar en forma sistemática las diferentes problemáticas de los usuarios, y diseñar productos o sistemas de software que queden expresados mediante algún lenguaje de programación, de acuerdo a los estándares de la ingeniería de software.}
    
    Esta competencia se pudo ver evidenciada por las diferentes etapas del levantamiento del servicio de condiciones comerciales: modelamiento, creación de un \textit{solver} y generación de vistas.
    
    Para la etapa de modelamiento se analizó la necesidad de los usuarios (gerentes, área contable y vendedores) de migrar las condiciones comerciales a un servicio especializado que tuviese un esquema y desarrollo flexible de manera de que abarcara todos los planes existentes y los futuros. Esta problemática fue identificada dado el tamaño de la empresa y el hecho de que las condiciones comerciales no habían logrado ser abarcadas de manera correcta en dos intentos anteriores y se resolvió con los esquemas planteados en la sección \ref{modelamiento}. Este paso quedó plasmado en el leguaje Ruby montado sobre el \textit{framework} Rails.

    Para la segunda etapa, se identificó la problemática del traspaso del cálculo de facturación mediante un archivo Excel no escalable como un problema por parte de la empresa y del área contable de esta. La solución a este problema se ve evidenciada mediante la creación de un \textit{solver} generalizado que se adapta automáticamente dependiendo de las condiciones comerciales de cada cliente y cada producto, quedando expresado en RoR con un esquema de delegación con patrones de diseño como se puede observar en la figuras de la sección \ref{solver_y_arquitectura}.

    Finalmente, para la tercera etapa, se hizo notar por parte de los gerentes la necesidad de acotar los campos editables por los vendedores (usuarios finales), por lo que el alumno, junto con Grace Lillo, diseñaron los formularios de creación de condiciones comerciales acorde a lo planteado. Adicionalmente, se realizaron validaciones con todas las partes involucradas, de manera de que se asegurara el resolver el problema de todos. Esta solución de formulario de creación de condiciones comerciales quedó expresada en el \textit{framework} React en el lenguaje JavaScript.

  \end{enumerate}

\section{Conclusiones generales}

  Acorde a lo trabajado durante cuatro meses por el alumno en Beetrack, se puede afirmar que se cumplieron las tres competencias planteadas inicialmente en la sección \ref{competencias} y que estas fueron ampliamente validadas. Para lograr las metas y competencias propuestas, el estudiante se desempeñó como Ingeniero de \textit{Software full-stack} dentro del área de operaciones en Beetrack.
  
  Los proyectos que el universitario llevó a cabo dentro del equipo de operaciones fuero dos: la creación de un servicio de condiciones comerciales  y el aumento sistemático de la cobertura de pruebas automatizadas. El servicio creado le permitirá a la empresa tener una facturación mucho más precisa que antes, logrando un 87,9\% de precisión, mientras que al mismo tiempo le facilitará el tener servicios especializados para el cálculo del costo por uso de cada cliente y producto y que este cálculo se integre de manera simple con el ERP. Por otro lado, el incremento de cobertura por parte de las pruebas automatizadas le entregará una mayor seguridad a al equipo y a \textit{Hive}, al mismo tiempo se podrá detener la puesta en producción del código fallido de manera preventiva gracias a la integración continua que se realizó con Circle CI.

  Finalmente, el alumno tuvo la oportunidad de desempeñarse de manera profesional y pudo utilizar de manera amplia los conocimientos adquiridos durante sus años universitarios. Adicionalmente, también logró desempeñarse de buena manera con diferentes áreas mediante un trabajo en equipo, lo que le permitió también hacer uso de las habilidades blandas adquiridas en la carrera. La combinación del uso de capacidades blandas como conocimientos teóricos y prácticos permitieron que el alumno lograra un buen trabajo en equipo y buen desempeño en cuanto a la resolución de problemas y soluciones innovadoras a problemas complejos.