\section{Competencias evidenciadas}

  El trabajo realizado por el alumno como ingeniero de \textit{software}, en Beetrack durante los cuatro meses evidencian de manera clara cada una de las tres competencias declaradas del perfil. En específico se pudo evidenciar lo siguiente:

  \begin{enumerate}
    \item Aplicar conocimientos avanzados de Ciencia de la Computación, Ingeniería de Software y Sistemas de Información para entender problemas complejos y abiertos.
    
    El estudiante desarrolló un sistema especializado con una arquitectura de software diseñada para ser escalable y replicable rápidamente mediante el uso de tecnologías como GKE, como también hizo uso avanzado de del \textit{frmework} RoR utilizando validadores personalizados en los modelos. Para poder abordar los dos puntos mencionados anteriormente, el estudiante tuvo que comprender las diferentes problemáticas del problema y utilizar sus conocimientos en las diferentes áreas para abarcarlos.

    Por otro lado, para poder diseñar de manera adecuada, tanto los aparatos de \textit{software} creados, el estudiante tuvo que aplicar conocimiento teórico de patrones avanazdos de diseño, de manera de que se pudiese tener un código altamente cohesivo y con bajo acoplamiento, algo que es sumamente deseado. Dentro de esta misma línea, el estudiante utilizó de manera amplia el \textit{framework} de pruebas RSpec, lo que implica un uso de buenas prácticas de \textit{software} que lo permite hacer más escalable, predecible y estable.

    Finalmente, el estudiante tuvo que utilizar conocimientos de sistemas de información, tales como la lógica de negocio de un ERP, que fue en lo que el alumno trabajó. Esto se traduce directamente en la combinación e integración que se realiza con Hubspot y Quickbooks, junto con el uso que \textit{Hive} le da a nivel de lógica de negocio.

    \item Diseñar y desarrollar modelos y artefactos de software y simular soluciones a problemas de la Ciencia e Ingeniería de Computación, cumpliendo con restricciones técnicas, sociales y éticas.

    \item Analizar en forma sistemática las diferentes problemáticas de los usuarios, y diseñar productos o sistemas de software que queden expresados mediante algún lenguaje de programación, de acuerdo a los estándares de la ingeniería de software.
  \end{enumerate}

\section{Conclusiones generales}

  