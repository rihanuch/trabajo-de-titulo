En el presente informe se presenta el trabajo de título realizado por Rodrigo Ignacio Hanuch González dentro de la empresa Beetrack S.A. para optar al grado de Ingeniero Civil de Industrias con diploma en Ingeniería de Computación. Durante cuatro meses el alumno trabajó como Ingeniero de \textit{Software full-stack} en el equipo de operaciones de la empresa. El estudiante desarrolló dos proyectos: el aumento sistemático de la cobertura del \textit{testing} y uso de este en la integración continua, junto con la creación de un servicio especializado para mantener y resolver las condiciones comerciales de los clientes.

El primer proyecto consistió en el aumento de la cobertura de las pruebas automatizadas del ERP de la empresa, junto con la integración de estas al flujo de integración continua en Circle CI. El objetivo principal de este proyecto fue el poder aumentar la seguridad del equipo al hacer cambios en el sistema de manera de que ante un cambio fallido las pruebas fallaran y se evitara poner en producción código potencialmente erróneo.

El segundo proyecto abarcó el problema de no tener un servicio y modelo dedicado al almacenamiento de condiciones comerciales flexibles, junto con un servicio de cálculo de costos para la generación de pre-facturas. Los principales objetivos del proyecto fueron la creación de un modelo flexible que pudiese abarcar las condiciones comerciales existentes, como las que se fueran a crear en un futuro, junto con un aumento en la precisión del cálculo de facturación para los clientes de manera de disminuir el error promedio durante la creación de las pre-facturas.

A través de los dos proyectos el alumno logró demostrar el cumplimiento de las tres competencias declaradas para obtener el grado de Ingeniero. El informe se divide en tres secciones, una primera donde se contextualiza la empresa y el trabajo, una segunda que detalla los proyectos en los cuales el estudiante trabajó y una última en donde se exponen las evidencias declaradas junto con conclusiones generales.

