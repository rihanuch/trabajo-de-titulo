\section{Objetivos y contextualización}

  Actualmente, la empresa tiene más de 700 clientes activos, cada uno con diferentes contratos, o condiciones comerciales, y se utiliza un archivo de Excel para mantener estas condiciones actualizadas. Al mismo tiempo, este excel también contiene la forma en que se calculan los costos totales por el uso del servicio para cada cliente. Similar a lo que ocurría con la falta de pruebas automatizadas, también resulta curioso el hecho de que Beetrack, siendo una empresa SaaS y teniendo su propio ERP, utilice esta plantilla para calcular cuánto cobrarle a los clientes. Es por lo anterior que decidió crear un servicio especializado en el almacenamiento y cálculo de estas condiciones comerciales, y cuyo único propósito fuese resolver el problema de cómo calcular los costos por cliente y poder almacenar los modelos que representaran los contratos generados con estos.
  
  Uno de los puntos más cruciales sobre este servicio y su contexto es el hecho de que ya se había intentado dos veces algo similar. El primer intento fue mediante un desarrollo interno en el ERP, el cual no tuvo los resultados esperados en términos de precisión al momento de facturar y se concluyó que no era responsabilidad de \textit{Hive} como tal tener esta información. El segundo intento fue la migración de estos planes a un proyecto que, en su minuto, se encontraba en fase \textit{beta} en Hubspot, en el cual se prometía poder tener la información siempre actualizada. Finalmente terminó ocurriendo lo mismo que en el primer intento, pero con peor precisión al momento de facturar de manera automática, entregando, en el mejor de los casos, una precisión de 43\%.

  Teniendo en consideración los requisitos y los resultados anteriores, los objetivos principales del proyecto fueron dos. El primer objetivo, fue construir un modelo lo suficientemente flexible como para poder abarcar los diferentes tipos de ventas que los vendedores realizan, tanto para productos como servicios. El segundo objetivo fue el programar un servicio de resolución comercial, o calculadora, que fuese, nuevamente, lo suficientemente flexible como para poder abarcar el modelo, pero que al mismo tiempo no necesitara información contextual más allá del uso para saber cuánto cobrarle a cada cliente, mientras que al mismo tiempo entregaba glosas prehechas por producto.

  El equipo de Operaciones decidió no acoplar este proyecto a \textit{Hive} de manera de evitar el cruce indebido de información y para así poder mantener las ideas aisladas y para así comenzar a tener una arquitectura de sistemas distribuidos en base a servicios. Lo anterior sumado al hecho de que el se prevee que, eventualmente, otros servicios y productos de la empresan, tales como el \textit{Customer Portal}, consuman los recursos expuestos por este servicio de manera de que los clientes también puedan saber cuánto se les cobrará o el nivel de costo que tienen para un día determinado del mes. Como se puede ver en la figura \ref{fig:arquitectura}, del capítulo \ref{metodologias}, la arquitectura que se decidió utilizar es una en la cual el servicio de condiciones está aislada del ERP, mientras que al mismo tiempo no limita a que otro servicio pueda consumir la información del servicio nuevo, lo cual hubiese ocurrido en caso de haber escrito el código en \textit{Hive} dada su naturaleza monolítica.

\section{Desarrollo}

\section{Resultados}