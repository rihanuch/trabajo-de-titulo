This scope of the present report is to present the work carried out by Rodrigo Ignacio Hanuch González within the company Beetrack S.A to obtain  the degree of Civil Industrial Engineer with a diploma in Computer Science. For four months the student worked as a Full-stack software Engineer in the company's operations team. The student developed two projects: the systematic increment of the coverage of the testing suite and their use in the continuous integration of the software and the creation of a specialized service to maintain the commercial conditions of the clients.

The first project consisted in increment of the coverage of the company's ERP testing suite, along with integration of these into the continuous integration flow in Circle CI. The main objective of this project was to be able to increase the reliability of the software when making changes to the codebase so that in the event of a failed code change, the tests would fail and potentially erroneous code would be prevented from being put into production.

The second project covered the problem of not having a dedicated service to the storage of flexible commercial conditions, together with a cost calculation service for the generation of invoice drafts. The main objectives of the project were the creation of a flexible model that could cover existing commercial conditions, as well as future conditions, together with an increment in the precision of the invoicing calculation for customers in order to reduce the average error than during the creation of invoice drafts.

Through the two projects presented, the student was able to demonstrate the three declared competencies to obtain the degree of Engineer. The report is divided into three sections, the first in which the company and work are contextualized, a second that details the projects in which the student worked and a last where that evidences how each declared competency is fulfilled, alongside general conclusions.