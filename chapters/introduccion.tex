\section{Contextualización de la empresa}
    
    Beetrack S.A. (de ahora en adelante, ``Beetrack'' o ``la empresa'') es una empresa que brinda un SaaS (\textit{Software as a Service}) dentro del rubro logístico, con un enfoque en la resolución del problema de última milla. Esta fue fundada el año 2013 por Sebastián Ojeda y Nicolás Kipreos como un emprendimiento con el fin de apoyar al creciente comercio electrónico en el proceso de seguimiento y despacho. El día de hoy, 8 años después, la empresa cuenta con más de 650 clientes activos y más de 100 empleados \cite{corporateit} entre Chile, Argentina, Perú, Colombia y México. Al mismo tiempo, esta se ha posicionado a si misma como uno de los líderes del mercado SaaS en América Latina por el éxito que ha tenido su principal producto, LastMile.
    
    Junto con LastMile, Beetrack también ofrece un SaaS de planificación, como lo es PlannerPro. Por un lado, LastMile permite tener un seguimiento en tiempo real de las entregas que se realizan, disminuyendo la incertidumbre de los clientes sobre el estado de su pedido. Por otro lado, PlannerPro es utilizado para planificar, optimizar y diseñar rutas de entregas para reducir los costos operacionales de reparto de la empresa que lo utilice.
    
    Actualmente la empresa es la más grande del rubro de SaaS de última milla en latinoamérica, siendo su competencia directa SimpliRoute y Drivin. Beetrack busca posicionarse como la mejor alternativa del mercado constantemente mediante la inclusión de nuevos productos y el servicio de calidad que entrega.
    
    Desde el punto de vista organizacional, la empresa se divide en múltiples áreas. Las anteriores son: administración, \textit{customer success}, \textit{payments}, desarrollo, recursos humanos, \textit{design ops}, \textit{marketing}, operaciones, producto, \textit{revenue operations}, ventas y \textit{data science}. Cada una de estas áreas trabaja en forma independiente con algunos cruces entre áreas en los que se tiene interacciones cruzadas entre distintos equipos. En particular, el trabajo realizado fue en el área de operaciones con interacciones con los equipos de ventas, \textit{design ops} y \textit{revenue operations}, entre otros.

    
\section{Descripción del proyecto}

    Los objetivos del proyecto son dos. En primer lugar, el llevar tecnología a todas las áreas de la empresa de manera de poder mejorar y facilitar los procesos internos. Dentro de esta misma idea, el segundo objetivo es poder lograr la mayor automatización posible en estos procesos de manera de que se cometan menos errores y que las diferentes áreas de la empresa puedan enfocarse realmente en lo que les compete por sobre la realización de procesos estandarizados.
    
    Para poder lograr los objetivos propuestos, la empresa decidió crear su propio ERP (\textit{Enterprise Resource Planning}) interno de manera de poder adaptar su \textit{software} a sus proceos y no sus procesos al \textit{software}. El nombre de este ERP es \textit{Hive}, el cual cuenta con diferentes partes y módulos, cada uno con el acceso restringido a los empleados correspondientes por área del módulo. Actualmente, el módulo más utilizado es el de \textit{Payments}, el cual corresponde al del área contable de la empresa que permite generar facturas a los diferentes clientes de Beetrack.
    
    Cabe mencionar que a pesar de que \textit{Hive} tiene un enfoque interno, este \textit{software} posee conexiones con, principalmente, 3 \textit{softwares} externos a este. En primer lugar, posee conexión con los productos de la empresa, esto es LastMile y PlannerPro. En segundo lugar el ERP está conectado con un Hubspot, el CRM (\textit{Customer Relationship Manager}) que Beetrack utiliza. En esta plataforma se encuentra toda la información relacionada a todos los clientes, como el estado de estos dentro de la empresa. Finalmente, \textit{Hive} también se encuentra conectado fuertemente con Quickbooks, un \textit{software} utilizado para llevar la contabilidad de la empresa, permitiendo generar facturas, como también conciliar pagos de clientes. En este sentido, \textit{Hive}, al estar conectado con estos sistemas externos, facilita la comunicación y automatización de procesos internos, tanto contables como de administración, de manera centralizada al agregar lógica de negocio a los diferentes procesos involucrados.


\section{Descripción de los trabajos realizados}

    El alumno ejerció como Ingeniero de \textit{Software Full Stack}, es decir, que estuvo a cargo tanto de las decisiones de diseño de las soluciones tecnológicas, como del desarrollo del \textit{backend}, que corresponde a la lógica de las aplicaciones, y del \textit{frontend}, que corresponde a la interfaz que el usuario final utiliza. Las principales tareas desarrolladas corresponden al análisis y modelamiento de esquemas de negocio y el análisis de espacios que pueden llevar a mejoras en el \textit{software}, junto con el desarrollo de las soluciones a los problemas encontrados.
    
    Por un lado, la empresa le propuso al alumno abordar el desafío de la reestructuración de la forma en que se estaba pensando y utilizando las condiciones comerciales de facturación de cada cliente en Hubspot. En particular se sugirió rehacer todo lo que se tenía desde cero, creando un \textit{software} especializado para esta tarea. Por otro lado, el alumno realizó su práctica profesional en la empresa y se dio cuenta durante ese período de tiempo que existía una falta de pruebas unitarias y de integración en el ERP de la empresa. Por el motivo anterior, el alumno le propuso a la empresa aumentar significativamente el nivel de cobertura de código de su ERP, junto con la integración de estas pruebas al flujo de integración continua en el \textit{software} Circle CI.


NO SE SI SEA REALMENTE NCESARIO HACER SECCIONES DE ESTO DADO QUE LA IDEA ES DESCRIBIRLO EN SUS PROPIAS SECCIONES EN ESPECIFICO.
% \subsection{\textit{Testing} e integración continua con Circle CI}
% \subsection{Servicio de condiciones comerciales}

\section{Competencias evidenciadas}

% 1.a Aplicar conocimientos avanzados de Ciencia de la Computación, Ingeniería de Software y Sistemas de
% Información para entender problemas complejos y abiertos
% 2.a Diseñar y desarrollar modelos y artefactos de software y simular soluciones a problemas de la Ciencia e
% Ingeniería de Computación, cumpliendo con restricciones técnicas, sociales y éticas.
% 2.b Analizar en forma sistemática las diferentes problemáticas de los usuarios, y diseñar productos o
% sistemas de software que queden expresados mediante algún lenguaje de programación, de acuerdo a los
% estándares de la ingeniería de software.

Mediante el trabajo realizado dentro de Beetrack, el alumno busca evidenciar las competencias del perfil de egreso. La primera competencia que se busca evidenciar es el ``\textbf{Aplicar conocimientos avanzados de Ciencia de la Computación, Ingeniería de Software y Sistemas de Información para entender problemas complejos y abiertos}''. Esta se vio ampliamente evidenciada mediante la implementación de las pruebas de integración en \textit{Hive}, junto con el la forma en que se desarrolló el servicio de condiciones comerciales. La segunda competencia corresponde a ``\textbf{Diseñar y desarrollar modelos y artefactos de software y simular soluciones a problemas de la Ciencia e Ingeniería de Computación, cumpliendo con restricciones técnicas, sociales y éticas}'', la cual se puede ver evidenciada por la implementación y subsecuentes pruebas realizadas con el servicio de condiciones comerciales para la facturación automática el mes de octubre junto con la simulación de ejecución de código en las pruebas unitarias creadas. En tercer, y último lugar, la competencia ``\textbf{Analizar en forma sistemática las diferentes problemáticas de los usuarios, y diseñar productos o sistemas de software que queden expresados mediante algún lenguaje de programación, de acuerdo a los estándares de la ingeniería de software}'' se pudo ver evidenciada mediante el proceso de diseño e implementación de tanto el \textit{frontend} en \textit{Hive} para la creación de las condiciones comerciales con el nuevo servicio creado, como el \textit{backend} de este servicio.